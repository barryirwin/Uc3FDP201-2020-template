%% define your degree by using the code CSE, DFO or ADS below
\documentclass[CSE]{nucthesis}
%% optimises document for double sided printing.
%\documentclass[twoside]{nucthesis} % adds 10mm extra for binding


%% CONFIGURATION
\newcommand{\myname}{A.N.Other}
\newcommand{\mytitle}{My super awesome title}

%% ======= BEGIN PACKAGES==
%% some of these packages may be useful for you
\usepackage{subfig}   % used for sub-figures
%\usepackage{lipsum}  %disable when implementing this generate space filling text
%\usepackage{todonotes} % https://ctan.org/pkg/todonotes?lang=en make notes for yourself and supervisor
\usepackage{wrapfig}  % allows text wrapping around figures
%\usepackage{tabularx} % tables with adjustable width columns
%\usepackage{tablefootnote}
%\usepackage{booktabs} % makes pretty tables (needs to be used explicitly) https://ctan.org/pkg/booktabs?lang=en

%\usepackage{lettrine} % BIG FIRST CHAPTER LETTER
%\usepackage{quotchap}

%% -------  START Title Page -------------
\title{\mytitle}
\author{\myname}
\date{\today}

\begin{document}

\maketitle
\titlepage
\thispagestyle{empty} 
%% -------  END Title Page -------------
%% -------  START Front matter ----------
\pagenumbering{roman}
\begin{abstract}
    
\end{abstract}
%\chapter*{ACM Classification}
%% This section includes a classification of your research domain as used by the ACM CCS - see https://www.acm.org/publications/class-2012
%% generate your code at: https://dl.acm.org/ccs and paste below
%\begin{CCSXML}
%<ccs2012>
%<concept>
%<concept_id>10010583</concept_id>
%<concept_desc>Hardware</concept_desc>
%<concept_significance>500</concept_significance>
%</concept>
%</ccs2012>
%\end{CCSXML}
%
%\ccsdesc[500]{Hardware}
%\printccsdesc

\chapter*{Acknowledgements}

%\noindent\rule[0.5ex]{1\columnwidth}{1pt}
\tableofcontents{}
\newpage
\listoffigures{}
\newpage
\listoftables{}
\newpage


%% -------  END Frontmatter ----------
%% -------  START Document Body ----------
\linespread{1.5} %% 1.5 spacing for document body
\pagenumbering{arabic}
%%% Add in body content or chapters here.
%%  a standard report structure has the following chapters.  
%%  IF you need to deviate from these you must discuss with your supervisor.
\chapter{Introduction}

\chapter{Literature Review}
%This is an example of using todonotes
%\todo{I need to write this!}
\chapter{Analysis/Implementation}
\begin{figure}[h!]%
    \centering
    \subfloat[sample1]{\includegraphics[width=3cm]{\NUCLogo } }%
    \subfloat[sample2]{\includegraphics[width=4cm]{\NUCLogo } }%
    \caption{sample figs}%
    \label{fig:samplefigs}%
\end{figure}
\chapter{Results}
\begin{figure}[h!]%
    \centering
    \subfloat[sample1]{\includegraphics[width=3cm]{\NUCLogo} }%
    \subfloat[sample2]{\includegraphics[width=4cm]{\NUCLogo} }%
    \caption{sample figs2}%
    \label{fig:samplefigs2}%
\end{figure}
\chapter{Conclusion}

\section{Future Work}
%% you could also keep each chapter in a separate file and include them here:
%\input{ch1/introduction}
%% This is recommended for making your life easier managing each chapter and any images related to it -- see the ch1 example directory.

%% ------ END Document Body ----------
%% -------  START References ----------
%% References
%% -------  END References ----------

%% -------  START Appendices ----------
%% appendices contain additional material and are numbered A, B, C etc beyond this point
%% you can use \chapter to differentiate them. Comment these out if not needed
\appendix
\chapter{Sample Appendix}
%% -------  END Appendices ----------
\end{document}
